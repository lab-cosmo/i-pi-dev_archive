\section{OUTPUTS}
\label{OUTPUTS}
\begin{ipifield}{}%
{This class defines how properties, trajectories and checkpoints should be output during the simulation. May contain zero, one or many instances of properties, trajectory or checkpoint tags, each giving instructions on how one output file should be created and managed.}%
{}%
{\ipiitem{prefix}%
{A string that will be prepended to each output file name. The file name is given by 'prefix'.'filename' + format\_specifier. The format specifier may also include a number if multiple similar files are output.}%
{default: `i-pi'; data type: string; }%
}
\begin{ipifield}{\hyperref[CHECKPOINT]{checkpoint}}%
{Each of the checkpoint tags specify how to create a checkpoint file, which can be used to restart a simulation. }%
{data type: integer; }%
{\ipiitem{stride}%
{The number of steps between successive writes.}%
{default:  1 ; data type: integer; }%
\ipiitem{overwrite}%
{This specifies whether or not each consecutive checkpoint file will overwrite the old one.}%
{default:  True ; data type: boolean; }%
\ipiitem{filename}%
{A string to specify the name of the file that is output. The file name is given by 'prefix'.'filename' + format\_specifier. The format specifier may also include a number if multiple similar files are output.}%
{default: `restart'; data type: string; }%
}
\end{ipifield}
\begin{ipifield}{\hyperref[TRAJECTORY]{trajectory}}%
{Each of the trajectory tags specify how to create a trajectory file, containing a list of per-atom coordinate properties. }%
{data type: string; }%
{\ipiitem{format}%
{The output file format.}%
{default: `xyz'; data type: string; options: `xyz', `pdb'; }%
\ipiitem{filename}%
{A string to specify the name of the file that is output. The file name is given by 'prefix'.'filename' + format\_specifier. The format specifier may also include a number if multiple similar files are output.}%
{default: `traj'; data type: string; }%
\ipiitem{stride}%
{The number of steps between successive writes.}%
{default:  1 ; data type: integer; }%
\ipiitem{flush}%
{How often should streams be flushed. 1 means each time, zero means never.}%
{default:  1 ; data type: integer; }%
\ipiitem{bead}%
{Print out only the specified bead. A negative value means print all.}%
{default:  -1 ; data type: integer; }%
\ipiitem{cell\_units}%
{The units for the cell dimensions.}%
{default: `'; data type: string; }%
}
\end{ipifield}
\begin{ipifield}{\hyperref[PROPERTIES]{properties}}%
{Each of the properties tags specify how to create a file in which one or more properties are written, one line per frame. }%
{data type: string; }%
{\ipiitem{stride}%
{The number of steps between successive writes.}%
{default:  1 ; data type: integer; }%
\ipiitem{shape}%
{The shape of the array.}%
{default:  (0,) ; data type: tuple; }%
\ipiitem{flush}%
{How often should streams be flushed. 1 means each time, zero means never.}%
{default:  1 ; data type: integer; }%
\ipiitem{filename}%
{A string to specify the name of the file that is output. The file name is given by 'prefix'.'filename' + format\_specifier. The format specifier may also include a number if multiple similar files are output.}%
{default: `out'; data type: string; }%
}
\end{ipifield}
\end{ipifield}
