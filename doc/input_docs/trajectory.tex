\section{TRAJECTORY}
\label{TRAJECTORY}
\begin{ipifield}{}%
{This class defines how one trajectory file should be output. Between each trajectory tag one string should be given, which specifies what data is to be output.}%
{data type: string; }%
{\ipiitem{format}%
{The output file format.}%
{default: `xyz'; data type: string; options: `xyz', `pdb'; }%
\ipiitem{filename}%
{A string to specify the name of the file that is output. The file name is given by 'prefix'.'filename' + format\_specifier. The format specifier may also include a number if multiple similar files are output.}%
{default: `traj'; data type: string; }%
\ipiitem{stride}%
{The number of steps between successive writes.}%
{default:  1 ; data type: integer; }%
\ipiitem{flush}%
{How often should streams be flushed. 1 means each time, zero means never.}%
{default:  1 ; data type: integer; }%
\ipiitem{bead}%
{Print out only the specified bead. A negative value means print all.}%
{default:  -1 ; data type: integer; }%
\ipiitem{cell\_units}%
{The units for the cell dimensions.}%
{default: `'; data type: string; }%
}
\end{ipifield}
