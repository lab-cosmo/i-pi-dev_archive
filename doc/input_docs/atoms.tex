\section{ATOMS}
\label{ATOMS}
\begin{ipifield}{}%
{Deals with a single replica of the system or classical simulations.}%
{}%
{}
\begin{ipifield}{q}%
{The positions of the atoms, in the format [x1, y1, z1, x2, \ldots  ].}%
{dimension: length; default:  [ ] ; data type: float; }%
{\ipiitem{units}%
{The units the input data is given in.}%
{default: `'; data type: string; }%
\ipiitem{shape}%
{The shape of the array.}%
{default:  (0,) ; data type: tuple; }%
}
\end{ipifield}
\begin{ipifield}{p}%
{The momenta of the atoms, in the format [px1, py1, pz1, px2, \ldots  ].}%
{dimension: momentum; default:  [ ] ; data type: float; }%
{\ipiitem{units}%
{The units the input data is given in.}%
{default: `'; data type: string; }%
\ipiitem{shape}%
{The shape of the array.}%
{default:  (0,) ; data type: tuple; }%
}
\end{ipifield}
\begin{ipifield}{natoms}%
{The number of atoms.}%
{default:  0 ; data type: integer; }%
{}
\end{ipifield}
\begin{ipifield}{m}%
{The masses of the atoms, in the format [m1, m2, \ldots  ].}%
{dimension: mass; default:  [ ] ; data type: float; }%
{\ipiitem{units}%
{The units the input data is given in.}%
{default: `'; data type: string; }%
\ipiitem{shape}%
{The shape of the array.}%
{default:  (0,) ; data type: tuple; }%
}
\end{ipifield}
\begin{ipifield}{names}%
{The names of the atoms, in the format [name1, name2, \ldots  ].}%
{default:  [ ] ; data type: string; }%
{\ipiitem{shape}%
{The shape of the array.}%
{default:  (0,) ; data type: tuple; }%
}
\end{ipifield}
\end{ipifield}
